\documentclass{gqtekspec}
%%%%%%%%%%%%%%%%%%%%%%%%%%%%%%%%%%%%%%%%%%%%%%%%%%%%%%%%%%%%%%%%%%%%%%%%%%%%%%%%
%%
%% Filename: 	spec.tex
%%
%% Project:	SD-Card controller, using a shared SPI interface
%%
%% Purpose:	This LaTeX file contains all of the documentation/description
%%		currently provided with the SDSPI controller core.  For those
%%	looking into here, this document is not nearly as interesting as the
%%	spec.pdf file it creates, so I'd strongly recommend reading that before
%%	diving into this document.  However, if you are interested in producing
%%	documents looking like this one, or using LaTeX in a similar fashion,
%%	you may find this document of use.
%%
%%	You should be able to find the PDF this file produces in the SVN
%%	distribution together with this LaTeX file and a copy of the GPL-3.0
%%	license this file is distributed under.  If not, just type 'make' in
%%	the doc directory (one up from this one), and it (should) build both
%%	pdf's without a problem.
%%
%% Creator:	Dan Gisselquist, Ph.D.
%%		Gisselquist Technology, LLC
%%
%%%%%%%%%%%%%%%%%%%%%%%%%%%%%%%%%%%%%%%%%%%%%%%%%%%%%%%%%%%%%%%%%%%%%%%%%%%%%%%%
%%
%% Copyright (C) 2016-2020, Gisselquist Technology, LLC
%%
%% This program is free software (firmware): you can redistribute it and/or
%% modify it under the terms of  the GNU General Public License as published
%% by the Free Software Foundation, either version 3 of the License, or (at
%% your option) any later version.
%%
%% This program is distributed in the hope that it will be useful, but WITHOUT
%% ANY WARRANTY; without even the implied warranty of MERCHANTIBILITY or
%% FITNESS FOR A PARTICULAR PURPOSE.  See the GNU General Public License
%% for more details.
%%
%% You should have received a copy of the GNU General Public License along
%% with this program.  (It's in the $(ROOT)/doc directory, run make with no
%% target there if the PDF file isn't present.)  If not, see
%% <http://www.gnu.org/licenses/> for a copy.
%%
%% License:	GPL, v3, as defined and found on www.gnu.org,
%%		http://www.gnu.org/licenses/gpl.html
%%
%%
%%%%%%%%%%%%%%%%%%%%%%%%%%%%%%%%%%%%%%%%%%%%%%%%%%%%%%%%%%%%%%%%%%%%%%%%%%%%%%%%
%%
%%
\usepackage{import}
\usepackage{bytefield}
\project{SDSPI Controller}
\title{Specification}
\author{Dan Gisselquist, Ph.D.}
\email{dgisselq (at) ieee.org}
%
% Revision level.  If you change this, be sure to change the revision history
% to match it
\revision{Rev.~0.31}
\begin{document}
\pagestyle{gqtekspecplain}
\titlepage
\begin{license}
Copyright (C) \theyear\today, Gisselquist Technology, LLC

This project is free software (firmware): you can redistribute it and/or
modify it under the terms of  the GNU General Public License as published
by the Free Software Foundation, either version 3 of the License, or (at
your option) any later version.

This program is distributed in the hope that it will be useful, but WITHOUT
ANY WARRANTY; without even the implied warranty of MERCHANTIBILITY or
FITNESS FOR A PARTICULAR PURPOSE.  See the GNU General Public License
for more details.

You should have received a copy of the GNU General Public License along
with this program.  If not, see \texttt{http://www.gnu.org/licenses/} for a copy.
\end{license}
\begin{revisionhistory}
% Any changes here need to also be made to the \revision{} in the prologue
0.31 & 11/29/2019 & Gisselquist & Fixed the bit mapping in Fig.~\ref{fig:CMD}\\\hline
0.3 & 11/05/2019 & Gisselquist & Major internal rewrite, formal properties\\\hline
0.2 & 10/21/2019 & Gisselquist & Card detect feature \\\hline
0.1 & 6/18/2016 & Gisselquist & First Draft \\\hline
\end{revisionhistory}
% Revision History
% Table of Contents, named Contents
\tableofcontents
\listoffigures
\listoftables
\begin{preface}
When I started this project, I was informed that other projects similar to this
one existed.  The OpenRISC project has used an SD--Card controller, for example,
as has the Google project vault.  Of these two, the first uses the full SD--Card
interface which is unavailable on the XuLA2 board, and I could never find the
code for the second.  

Still, had I found such interfaces, I would've still had another reason for
building my own: controlling the license.  By rolling my own interface, I can
offer it to anyone interested in it under the GPL license, such as you have
here.  Further, by not using code belonging to others, I am not restricted or
encumbered by any of their licenses--whether it be the GPL or otherwise.  This
code, and specification document, are therefore completely the product of
Gisselquist Technology, LLC.

This particular core also maintains an advantage over the OpenRISC core:
It is a low logic core.  It only supports the SPI interface.  It does not
have any DMA features, although it will work nicely with a DMA.  In short, it's
ideal for other low-logic work.  In that line, I think, this core has found
its niche.
\end{preface}

\chapter{Introduction}
\pagenumbering{arabic}
\setcounter{page}{1}

% What is old
This Verilog core exports an SD card controller interface from internal to an
FPGA to the rest of the FPGA core, while taking care of the lower level details
internal to the interface.
% What does the old lack?
Unlike the other OpenCores SD Card controller\footnote{See
\texttt{http://www.opencores.org/project,sdcard\_mass\_storage\_controller}.}
which offers a full SD--interface, this controller focuses on the SPI interface
of the SD Card.  While this is a slower interface, the SPI interface is
necessary to access the card when using a XuLA2
board\footnote{See \texttt{http://www.xess.com/shop/product/xula2-lx25/}}, or
in general any time the full 9--bit, bi--directional interface to the SD card
has not been implemented.
Further, for those who are die--hard Verilog authors, this core is written in
Verilog as opposed to the XESS provided demonstration SD Card controller
found on GitHub\footnote{See
\texttt{https://github.com/xesscorp/VHDL\_Lib/SDCard.vhd}}, which was written
in VHDL.  For those who are not such die--hard Verilog authors, this controller
provides a lower level interface to the card than these other controllers. 
Whereas the XESS controller will automatically start up the card and interact
with it, this controller requires external software to be used when interacting
with the card.  This makes the SDSPI controller both more versatile, in the
face of potential changes to the card interface, but also less turn--key.
% What is new
% What does the new have that the old lacks
% What performance gain can be expected?

While this core was written for the purpose of being used with the ZipCPU,
as enhanced by the Wishbone DMA controller used by the ZipCPU, nothing in this
core prevents it from being used with any other architecture that supports
the 32--bit Wishbone interface of this core.

This core has been written as a wishbone slave, not a master.  Using the core
together with a separate master, such as a CPU or a DMA controller, only makes
sense.  This design, however, also restricts the core from being able to use
the multiple block write or multiple block read commands, restricting us to 
single block read and write commands alone.

\chapter{Architecture}\label{ch:arch}

This SD Card interface is designed to provide a means of commanding an SD Card,
via the SPI port, and returning its results.  

It is completely controlled via the wishbone bus.  In
particular a command register is used to initiate interaction across the bus.
A separate data register is used to provide an argument to the command, and
two FIFO registers are used when transferring larger amounts of data to the
card.  We'll examine each of these interactions in turn.

Writes to the command register (CMD) may initiate actions across the port,
whether they be reads from or writes to the card.  These writes take the
form of sending a 48--bit command to the card.  The command sent to the card
is taken from the lower 8--bits of the command register, and the argument to the
command is taken from the DATA register.  The last 8--bits of the command sent
to the card are formed from a command CRC byte which the core generates
internally.  From the perspective of the Wishbone bus, writes to the register
will complete immediately, even though the action they initiate will may take
much longer to complete.  Further, writes made to the CMD register will be
silently ignored if the device is already busy.

Reads from the CMD register will always return immediately.  In particular, the
{\tt busy} bit, as returned by the CMD register, can be used to determine if the
interface is still busy with a prior operation.

There is one exception to the rule that writes take many clocks to complete,
and that is writes which configure the SDSPI port.  Internal to the SDSPI
port is a configuration register, which determines the speed of the port
clock as well as the length of the FIFO (up to 128~samples, or 512--bytes).
To read the current speed and FIFO configuration, write an {\tt 0x00bf} value to
the command register.  This will cause the DATA register to be filled with
the internal configuration register.  Likewise writing a {\tt 0x0ff} to the
CMD register will cause the current DATA value, or specifically those non-zero
parts, to be transferred to the internal configuration register possibly setting
the clock divider and/or the FIFO length.

As part of each write to the CMD register, the controller must also be told
which type of response to expect from the SDSPI card.  Responses can be either
R1 (single byte), R1b (single byte, followed by a variable delay), or R1
followed by up to four bytes, such as the R2, R3, or R7 responses.
(Expected responses for particular commands may be found in the SD
Specifications documents.\footnote{This particular interface, and the examples
using it, were built using the SD Specifications, Part 1: Physical Layer
Simplified Specification, Version 4.10, dated 22 January, 2013, and then later
updated with the information from Version 5.00, dated 10 August, 2016.}

Individual commands may or may not use the data memories, herein called
FIFOs.  Commands that need
use of the FIFO will be specified by the {\tt use\_fifo} bit of the CMD
register.  Commands writing to the card will also set the {\tt fifo\_wr} bit
of the CMD register, whereas commands simply reading from the fifo will set the
{\tt use\_fifo} bit alone.

The DATA register is used during these transactions to first provide the 
argument to the CMD interaction, and second to provide a place to put the
R2, R3, or R7 response after the transaction has completed.  The register will
be set to {\tt 0xffffffff} if not set by the response.

This core supports two separate data memories.  This allows a program to
fill (or read) one memory segment while the second one is being read from or
written to by an ongoing SD operation.
The internal address will be cleared and reset to the beginning
upon any write to the CMD register.  After clearing, the FIFO may be written
(read) one value at a time.  Reading both memories in any interleaved fashion,
however, is not allowed as they share a common internal address.

Currently, the core will detect a variety of errors in the interface.  Once an
error is detected, the rest of any remaining command will be aborted.  First,
the core will detect an external Card reset.  Such a reset is used on the
NexysVideo board to power down the card.  Second, the core will detect any
CRC errors in data coming from the card.  Finally, the core has an internal
watchdog timer and will detect any failure by the card to respond to any
request.  Any of these errors will set an error bit in the CMD register.
Once set, the core will refuse to begin further operations.  only writing
this error bit back to the CMD register will clear it.

Finally, if a card detect bit is present, the core can detect if a card has
been removed and so notify the driver.  A missing card detect signal, however,
will not reset the core.  Instead, the watchdog timer should catch any missing
card interactions.  The card detect interface was added to simplify
FatFS driver interaction.

Now, if this discussion isn't thoroughly confusing, let's move on to the 
Operation chapter to see some examples of how this might be used.

\chapter{Operation}\label{ch:ops}
This chapter will walk through some constants that can be used to simplify
interaction with the controller, the logic necessary to start up the card,
to read its registers, and then examples of how to read and write sectors
from the SD Card using this interface.

\section{Constants}
Since so much of the interface is controlled by the CMD register, it helps to
define several constants which can be used when issuing commands to the SD
Card.  Lets discuss some of these constants.

First, as discussed in the last chapter, the SDSPI core maintains an auxiliary
register to handle FIFO length and clock speed.  To set this register, we
define {\tt SD\_SETAUX} to {\tt 0x0ff}.  Thus, when {\tt SD\_SETAUX} is written
to the CMD register, the value of the DATA register is transferred to the
internal configuration.  Likewise, we also define {\tt SD\_READAUX} to
{\tt 0x0bf}.  When this value is written to the SD--Card, the internal
configuration registers value will be copied to the DATA regiseter.
\begin{tabular}{lll}
{\tt \#define} & {\tt SD\_SETAUX} & {\tt 0x0ff} \\
{\tt \#define} & {\tt SD\_READAUX}  & {\tt 0x0bf}
\end{tabular}

Second, every command to the SD--Card starts with a single byte.  Of that byte,
bit-7 must be clear and bit 6 set.  For this purpose, we define {\tt SD\_CMD}
to be {\tt 0x040}.  Thus, {\tt SD\_CMD+0} can be used to send an SD command
{\tt CMD0}, and {\tt SD\_CMD+1} can be used to send an SD command {\tt CMD1}.
\begin{tabular}{lll}
{\tt \#define} & {\tt SD\_CMD} & {\tt 0x040}
\end{tabular}

Third, for those commands that will read an SD--Card register, such as those
expecting an R2, R3, or R7 response from the card, we define {\tt SD\_READREG}
to be {\tt 0x0200}.  Thus, we can send a CMD8 by writing
\hbox{\tt SD\_CMD|SD\_READREG} to the port.
\begin{tabular}{lll}
{\tt \#define} & {\tt SD\_READREG} & {\tt 0x0200}
\end{tabular}

The next thing we'll want to be able to do is use the FIFO.  There are two
types of commands that use the FIFO, those that read from the card and those
that write to the card.  Both need the FIFO bit set, so we'll set
{\tt SD\_FIFO\_OP} to {\tt 0x0800} to be a read operation from the card, and
the same but with the write bit set {\tt SD\_WRITEOP} will be set to
{\tt 0x0c00} to write to the card.

\begin{tabular}{lll}
{\tt \#define} & {\tt SD\_FIFO\_OP} & {\tt 0x800} \\
{\tt \#define} & {\tt SD\_WRITEOP}  & {\tt 0xc00}
\end{tabular}

Finally, we want to be able to choose which FIFO we are using.  For this
purpose, we define {\tt SD\_ALTFIFO} to be {\tt 0x01000}.  When this bitmask
is included in a command, FIFO number one will be used for the command data,
otherwise FIFO zero.  (Note that this is separate from the DATA register,
which is still used for any command argument.)

\begin{tabular}{lll}
{\tt \#define} & {\tt SD\_ALTFIFO} & {\tt 0x1000}
\end{tabular}

Two other constants are necessary: {\tt SD\_BUSY}, set to {\tt 0x04000}, which
can be used to test when the SD interface is still busy, and {\tt SD\_ERROR},
set to {\tt 0x08000} which can be used to tell if an error has occurred.
Clearing an error may be done by writing {\tt SD\_ERROR} back to the card, but
to make things simpler we also create {\tt SD\_CLEARERR} for the same purpose.

\begin{tabular}{lll}
{\tt \#define} & {\tt SD\_BUSY}     & {\tt 0x04000} \\
{\tt \#define} & {\tt SD\_ERROR}    & {\tt 0x08000} \\
{\tt \#define} & {\tt SD\_CLEARERR} & {\tt 0x08000}
\end{tabular}

The controller offers two means of knowing whether or not the card is present.
The first is a {\tt SD\_PRESENTN} bit.  This is a debounced version of the
card detect input, adjusted so that it will be set if no card is present--to
make error detection easier.  If this bit is clear (normal
operation), then a card is present and ready to be set up.  On the other hand,
if the card detect input ever goes low then the second bit, {\tt SD\_REMOVED},
will be set.  Unlike the {\tt SD\_PRESENTN} bit which shows the current state
of whether a card is present or not, the {\tt SD\_REMOVED} bit is sticky.
Once set, it will remain set until explicitly written to.  This allows us to
clear it on an {\tt SD\_GO\_IDLE} command, and
otherwise leave it alone.  If the {\tt SD\_REMOVED} flag ever goes high, then
the driver knows it's time to restart the interface with a new card.
\begin{tabular}{lll}
{\tt \#define} & {\tt SD\_REMOVED}  & {\tt 0x40000} \\
{\tt \#define} & {\tt SD\_PRESENTN} & {\tt 0x80000}
\end{tabular}

{\tt SD\_GO\_IDLE} is an abbreviation for command zero, but in a starting over
context.  Not only does it send the command zero, but it will also clear any
unacknowledged errors and clear the {\tt SD\_REMOVED} bit.  After this command,
therefore, if the {\tt SD\_REMOVED} bit ever goes high the protocol will need
to start over with another {\tt SD\_GO\_IDLE} command.
\begin{tabular}{lll}
{\tt \#define} & {\tt SD\_GO\_IDLE} & {\tt ((SD\_REMOVED|SD\_CLEARERR|SD\_CMD)+0)}
\end{tabular}

The two most important commands, though, are probably going to be those that
read and write a sector.  For these, we shall define {\tt SD\_READ\_SECTOR}
and {\tt SD\_WRITE\_SECTOR}.  As the first is a CMD17 to the card and the second
a CMD24, these can be defined as:

\begin{tabular}{lll}
{\tt \#define} & {\tt SD\_READ\_SECTOR} & {\tt ((SD\_CMD|SD\_CLEARERR|SD\_FIFO\_OP)+17)} \\
{\tt \#define} & {\tt SD\_WRITE\_SECTOR} & {\tt ((SD\_CMD|SD\_CLEARERR|SD\_WRITEOP)+24)}
\end{tabular}

`Or'ing the {\tt SD\_ALTFIFO} mask to either of these commands will cause the
interface to read from or write to the alternate FIFO.

As a very last \#define, we can define the macro {\tt SD\_WAIT\_WHILE\_BUSY}
to wait until the SD operation completes:

\begin{tabular}{lll}
{\tt \#define} & {\tt SD\_WAIT\_WHILE\_BUSY}&{\tt while(CMD \& SD\_BUSY)}
\end{tabular}

Alternatively, we could wait for an interrupt instead since the SDSPI core
will create an interrupt upon completion.  For now, and for this example,
we'll ignore interrupts.

\section{SD--Card Setup}
Setting up an SD--Card takes a bit of work.  There's a series of commands 
and interactions that need to take place with the card before the card can
be used.  You can read about how to do this within the SD--Specification, so
we won't repeat the how's or why's here.  Instead, let's focus for now on 
how this interaction can be made to take place using this controller.

The first step in any start up sequence is to clear the card from any
prior condition.  Hence we wait for the card to be no longer busy (it 
shouldn't be busy anyway), and we then clear any errors:
\begin{tabbing}
{\tt SD\_WAIT\_WHILE\_BUSY;} \\
{\tt CMD} \= {\tt = SD\_CLEARERR};
\end{tabbing}

Now that the controller is idle (which it should've been from startup anyway),
we can now set up our interface.  For this, we'll set our clock rate to 400~KHz.
The clock division register, sometimes erroneously called the speed, is found
in the lower sixteen bits of the soft-core configuration register.  The actual
SPI clock frequency, given this value, will be:
\begin{eqnarray}
f_{\mbox{\tiny SDSPI}} &=& \frac{f_{\mbox{\tiny CLK}}}{2\left(\mbox{\tt CLKDIV}+1\right)}
\end{eqnarray}
where $f_{\mbox{\tiny CLK}}$ is the rate of the syste clock provided to the
core.  Hence, since the XuLA2-LX25 SoC runs at an 80~MHz clock, setting this
value to {\tt 0x63} sets the SPI clock to 400~kHz.
\begin{tabbing}
{\tt DATA} \= {\tt = 0x063}; \\
{\tt CMD} \> {\tt = SD\_SETAUX};
\end{tabbing}
Note that we could have also set the higher order configuration bits to set
the size of the FIFO.  In particular, the next four bits, bits 16--19, set
the block length.  Setting these to zero will cause the controller to ignore
the change, whereas setting the value to three will set the FIFO length to
$2^3$ bytes, and setting it to nine will set the FIFO length to the
nominal $2^9$ or $512$~bytes.

The controller is now ready to send commands to the SD card.  The first command
to the card is always a command zero, with zero data.  This is sometimes called
the \hbox{\tt GO\_IDLE\_STATE} command.  We then wait for the command to
complete:
\begin{tabbing}
{\tt DATA} \= {\tt = 0;} \\
{\tt CMD} \> {\tt = SD\_GO\_IDLE;} \\
{\tt SD\_WAIT\_WHILE\_BUSY;}
\end{tabbing}
This will also clear any card-inserted {\tt SD\_REMOVED} flag.  Once complete,
the card should now be in its idle state.

This is also the first command that might have an error.  In particular,
{\tt SD\_ERR} will be set if the card does not respond to the command.

Some specifications require a CMD1, \hbox{\tt SEND\_OP\_COND}, to be
sent next.  This is to tell the card whether or not high capacity is supported.
For this, we send a command one,
\hbox{\tt SEND\_OP\_COND}, with an argument of {\tt 0x40000000} to tell it
that we are able to support high capacity cards.  (An argument of zero would
mean that we could not.)
\begin{tabbing}
{\tt DATA} \= {\tt = 0x40000000;} \\
{\tt CMD} \> {\tt = SD\_CMD+1;} \\
{\tt SD\_WAIT\_WHILE\_BUSY;}
\end{tabbing}

Not all cards require or accept a CMD1 anymore, and indeed cards in my most
recent tests would stop responding if given a CMD1.  Further, it appears to have
been removed from the most recent SD-Card specification.

The card then needs to know what voltage it will be run at.  We communicate this
via a \hbox{\tt SEND\_IF\_COND} command, or CMD8.  Since most FPGA boards offer
only fixed 3.3V I/O configurations, we tell the card we wish to run at 3.3V in
the argument.  The last eight bits of the argument, however, are simply to
determine whether communication has taken place.  We set these bits to
{\tt 0x0a5}, although they could be anything.  The card will echo this value
back in the response:
\begin{tabbing}
{\tt DATA} \= {\tt = 0x1a5;} \\
{\tt CMD} \> {\tt = SD\_CMD+8;} \\
{\tt SD\_WAIT\_WHILE\_BUSY;} \\
{\em // assert(DATA == 0x01a5);}
\end{tabbing}
The card will also echo back the voltage range, if it accepts it.  Thus,
we should receive {\tt 0x01a5} as a response.

The card will now try to start up its own internal state machines.  This could
take a while.  We therefore poll the device, and wait for its startup sequence
to complete:
\begin{tabbing}
{\tt bool dev\_busy = false;} \\
{\tt do \{}\=\\
\> {\em // CMD55 gives us access to SD specific commands}\\
\> {\tt DATA = 0;}\\
\> {\tt CMD = SD\_CMD+55;}\\
\> {\tt SD\_WAIT\_WHILE\_BUSY;}\\
\\
\> {\em // Now we can issue the ACMD41, to get the idle}\\
\> {\em // status}\\
\> {\tt DATA = 0x40000000;} \\
\> {\tt CMD = SD\_CMD+41;} \\
\> {\tt DATA} \= {\tt = 0x1a5;} \\
\> {\tt CMD} \> {\tt = SD\_CMD+8;} \\
\> {\tt SD\_WAIT\_WHILE\_BUSY;} \\
\\
\> {\em // The R1 response can be found in the lower 8 bits}\\
\> {\em // of the CMD register after the command is complete.}\\
\> {\em // Bit 1 of R1 indicates the card hasn't finished its}\\
\> {\em // startup}\\
\> {\tt dev\_busy = CMD\&1;}\\
{\tt \} while(dev\_busy);}
\end{tabbing}

\section{Reading Card Registers}

Once the card has started, we can request its operating conditions register,
or OCR register as it is called.  For this, we issue a {\tt READ\_OCR} command,
or CMD58 by number.  Since this command returns a 32--bit value, we use the
{\tt SD\_READREG} macro as well:
\begin{tabbing}
{\tt int OCR;}\\
\\
{\tt DATA} \= {\tt = 0;} \\
{\tt CMD} \> {\tt = (SD\_READREG|SD\_CMD)+58;} \\
{\tt SD\_WAIT\_WHILE\_BUSY;} \\
{\tt OCR} \= {\tt = DATA;}\\
\end{tabbing}
When I issue this command on my card, I get a {\tt 0xc0ff8000} response telling
me that my card can handle between 2.7 and 3.6~Volts, that it is a higher
capacity card, and that it has completed its startup sequence.

Now let's switch up to a higher speed, and read the 16--byte Card Specific
Data (CSD) register field from the card.  First, the switch to a 20~MHz clock
and a 16--byte fifo,
\begin{tabbing}
{\tt DATA} \= {\tt = 0x040001}; \\
{\tt CMD} \> {\tt = SD\_SETAUX};
\end{tabbing}
Remember that the {\tt 0x040000} switches to a memory length of $2^4$ bytes.
Likewise the {\tt 0x0001} component of the configuration word switches
our frequency to $f_{\mbox{\tiny CLK}}/4$ or 20~MHz if starting with an
$80$~MHz clock.  Now we can issue the {\tt SEND\_CSD\_COND}, or
CMD9, command itself.  Note that we didn't need to wait for the
{\tt SD\_SETAUX} command to complete.  Further, since this command is going
to read from the SD card into our internal memory, we also need to include
the {\tt SD\_FIFO\_OP} part of the command:
\begin{tabbing}
{\tt int CSD[4];}\\
{\tt DATA} \= {\tt = 0;} \\
{\tt CMD} \> {\tt = (SD\_FIFO\_OP|SD\_CMD)+9;} \\
{\tt SD\_WAIT\_WHILE\_BUSY;} \\
{\tt for(int i=0; i<4; i++) } \\
\> {\tt CSD[i] = FIFO[0];}
\end{tabbing}
Once the command is complete, we can read the four 32--bit words of the CSD
register from the memory area, as shown above.  Alternatively, we could have
issued another command first, before reading that FIFO result.

We could also read the Card Identification (CID) register if we wanted as well.
Doing so would require the same sequence as above, save only that we would've
written a 
{\tt (SD\_READREG|SD\_CMD)+10} to CMD.

Reading the STATUS is similar, only the response to the {\tt SEND\_STATUS}
command is an 8--bit value from an R2 response, not the 32--bit values of
the R3 (OCR) or R7 responses.  The core will still read 32--bits, however,
and so it will place the R2 response in the upper 32-bits of the status word.
It's still provided in the DATA register, so we only need to send
{\tt (SD\_READREG|SD\_CMD)+13} to
the CMD register in order to read its result from the DATA register.  The status
register will be returned in the top eight bits of the DATA register (the
interface still reads 32--bits, even though the other 24 can be ignored), so:
\begin{tabbing}
{\tt int card\_status;}\\
{\tt DATA} \= {\tt = 0;} \\
{\tt CMD} \> {\tt = (SD\_READREG|SD\_CMD)+13;}
{\tt SD\_WAIT\_WHILE\_BUSY;} \\
{\tt card\_status = DATA>>24;}
\end{tabbing}

As a final register example, let's read the SD Card Configuration Register
(SCR).  This register is read in a fashion very similar to the CSD register,
except that because of its width the FIFO needs to be set for a shorter
register width:
\begin{tabbing}
{\tt int SCR[2];}\\
{\em // Set the FIFO length to 8 bytes, or $2^3$.}\\
{\tt DATA} \= {\tt = 0x030000}; \\
{\tt CMD} \> {\tt = SD\_SETAUX};\\
{\em // Issue an ALT command, to get the other command set.}\\
{\tt DATA} \= {\tt = 0;} \\
{\tt CMD} \> {\tt = (SD\_CMD)+55;}\\
{\tt SD\_WAIT\_WHILE\_BUSY;} \\
{\em // Now get the SCR register.}\\
{\tt DATA} \= {\tt = 0;} \\
{\tt CMD} \> {\tt = (SD\_FIFO\_OP|SD\_CMD)+51;}\\
{\tt SD\_WAIT\_WHILE\_BUSY;} \\
{\tt for(int i=0; i<2; i++) } \\
\> {\tt SCR[i] = FIFO[0];}\\
\end{tabbing}


\section{Reading and Writing}

For our first example, let's read the boot sector from our card.  For this,
we set our FIFO back to 512~bytes, and then issue a read sector command:
\begin{tabbing}
{\tt void} \= {\tt read(int sector\_num, int *buf) \{}\\
\> {\em // Set the FIFO length to 512 bytes, $2^9$.}\\
\> {\tt DATA} \= {\tt = 0x090000}; \\
\> {\tt CMD} \> {\tt = SD\_SETAUX};\\
\> {\em // Read from the requested sector}\\
\> {\tt DATA} \= {\tt = sector\_num;} \\
\> {\tt CMD} \> {\tt = SD\_READ\_SECTOR;}\\
\> {\tt SD\_WAIT\_WHILE\_BUSY;} \\
\> {\tt for(int i=0; i<512/4; i++) } \\
\> \> {\tt buf[i] = FIFO[0];}\\
{\tt \}}
\end{tabbing}

We could also write to any sector on the card in a very similar fashion:
\begin{tabbing}
{\tt void} \= {\tt write(int sector\_num, int *buf) \{}\\
\> {\em // Set the FIFO length to 512 words, $2^9$.}\\
\> {\tt DATA} \= {\tt = 0x090000}; \\
\> {\tt CMD} \> {\tt = SD\_SETAUX};\\
\> {\em // Fill the FIFO with our data}\\
\> {\tt for(int i=0; i<512/4; i++) } \\
\> \> {\tt FIFO[0] = buf[i];}\\
\> {\em // Issue the write command}\\
\> {\tt DATA} \= {\tt = sector\_num;} \\
\> {\tt CMD} \> {\tt = SD\_WRITE\_SECTOR;}\\
\> {\tt SD\_WAIT\_WHILE\_BUSY;} \\
{\tt \}}
\end{tabbing}

As mentioned in the introductory chapter, this interface does not support
reading or writing multiple blocks at once.  Hence, I expect all interaction
using this card controller to be accomplished through these two commands:
reading a single sector, and writing to a single sector.

\chapter{Registers}\label{ch:regs}

As mentioned in the last two chapters, the SDSPI core has only four registers,
and one internal register.  These are shown in Tbl.~\ref{tbl:ioregs}.
\begin{table}[htbp]
\begin{center}\begin{reglist}
CMD     &{\tt 0x00} & 32 & R/W & SDSPI Command and status register\\\hline
DAT     &{\tt 0x01} & 32 & R/W & SDSPI return data/argument register\\\hline
FIFO[0] &{\tt 0x02} & 32 & R/W & FIFO[0] data\\\hline
FIFO[1] &{\tt 0x03} & 32 & R/W & FIFO[1] data\\\hline
CONFIG  &  & 12 & R/W & Internal configuration register\\\hline
\end{reglist}
\caption{I/O Peripheral Registers}\label{tbl:ioregs}
\end{center}\end{table}
The most powerful of these is the command register, CMD, so we'll spend most of
our time discussing that one.

\section{CMD Register}
Writes to the CMD register will cause the device to act, or if the device is
already busy then any writes will be ignored.  The CMD register itself is
composed of several packed bit fields, as shown in Fig.~\ref{fig:CMD}.
\begin{figure}\begin{center}
\begin{bytefield}[endianness=big]{32}
\bitheader{0-31}\\
\bitbox{12}{Unused}
\bitbox{1}{P}	% PRESENTn,0x80000
\bitbox{1}{R}	% REMOVED, 0x40000
\bitbox{2}{}	% Reserved,  0x20000
\bitbox{1}{E}	% ERROR,   0x08000
\bitbox{1}{B}	% BUSY,    0x04000
\bitbox{1}{0}	% Unused
\bitbox{1}{I}	% 0x01000
\bitbox{1}{F}
\bitbox{1}{W}
\bitbox{2}{R}
\bitbox{8}{R1/CMD}
\end{bytefield}
\caption{CMD Register fields}\label{fig:CMD}
\end{center}\end{figure}
Perhaps the most important of these is the R1/CMD field.  On any write, if
bits 7--6 are the two bits 2'b01 and if the card is idle, then the command
contained in the rest of the R1/CMD field is sent to the card.  Once the
command is complete, these 8--bits represent the R1 response from the device.
According to the SD specification, R1 should be one while the device is still
starting, or zero in the case of no error.  Further interpretation of this
value may be found in the SD--Card Specification.

Of next importance is the $R$ field.  This specifies the response the
controller should expect from the card given the command that was issued
to the card.  There are three possible values for this field: $2'b00$, meaning
the controller should expect an R1 response, $2'b01$, meaning the controller 
should expect an R1b response, and $2'b10$ meaning the controller should
expect an R2/R3/R7 32--bit response.

The $F$, or FIFO, field should be set if the command being given requires a
data transmission to accompany it, either coming from or going to the internal
memories (FIFOs).  $W$ should be set at the same time if the controller will be
writing to the card from the FIFO, and cleared if the controller will be
reading from the card into the FIFO.  Finally, $I$ specifies which data memory
will be used: 0 for the primary, or 1 for the alternate.

While the command is running, the BUSY or $B$ bit will be set.

Once the command has completed, the $E$ bit may be set if either the command 
timed out, a card reset was received, or a CRC error was noted while reading
from the card.  It will also be set if the R1 response indicates an error
of some type has occurred, such as a CRC error when writing to the card.
Errors may be cleared by writing a 1 to the $E$ bit.
Error conditions will persist until cleared.  While an error condition is
present, the data memories sub-components be held in reset preventing any
data transactions.

The $R$ register is used to detect if a card is ever removed.  This bit is
cleared by writing a `1' to it--typically as part of the {\tt SD\_GO\_IDLE}
command.

Finally, the $P$ register can be used to determine if a card is present at all.
If $P$ is high, no card is present.  If $P$ is low, a card is present.  If
$P$ is low but $R$ is high, then a card has been inserted since the last
{\tt SD\_GO\_IDLE} command, and the new card should be initialized.

\section{DATA Register}
Compared to the CMD register, the DATA register is quite simple.  Like the
CMD register, the DATA register may only be written when the interface is
idle.  When issuing a command to the device, the 32--bit argument for the
command is taken from the DATA register.  When reading the results of a device
command, the DATA register will contain the R2 response in the upper 8--bits,
or any R3 or R7 response in the full 32--bits.  Following a memory block write
command, the DATA register will contain the token acknowledging the command.

\section{FIFO Registers}
The SDSPI controller maintains two 128~word (512~byte) memory areas called
FIFOs.  Reads from the card will write data into one of the two FIFO's, whereas
writes to the card will read data out from one of the FIFO's.  Which FIFO the
card uses is determined by the $I$ bit in the CMD register (above).

Further, upon any write to the CMD register, the FIFO address will be set
to point to the beginning of the FIFO.

The purpose of the FIFO's is to allow one to issue a command to read into one
FIFO, then when that command is complete to read into a second FIFO.  While
the second command is ongoing, a CPU or DMA may read the data out of the first
FIFO and place it wherever into memory.  Then, when the second read is complete,
a third read may be issued into the first buffer while the data is read out of
the second and so forth.

This interleaving approach, sometimes called ping-pong buffering, can also be
used for writing: Write into one FIFO, issue a write command, write into the
second FIFO, wait for the first write command to complete, issue a second
write command, and so forth.

One item to note before closing: there is only one internal address register
when accessing the FIFO from the wishbone bus.  Attempts to read from or write
to either FIFO from the wishbone bus will increment this address register. 
Interleaved read, or write attempts, such as reading one item from
FIFO[0] and writing another item to FIFO[1], will each increment the internal
address pointer so that the result is likely to be undesirable.  For this
reason, it is recommended that only one FIFO be read from or written by the
wishbone bus at a time.

\section{CONFIG Register}
The CONFIG register controls the SPI clock rate and the FIFO size. 
Specifically, with regards to the FIFO size, it controls how many bytes will
be written into the FIFO (which is really of a fixed size) before the expecting
a CRC, or equivalently how many bytes to read out of the FIFO before adding a
CRC.  The fields of this register are shown in Fig.~\ref{fig:CONFIG}.
\begin{figure}\begin{center}
\begin{bytefield}[endianness=big]{32}
\bitheader{0-31}\\
\bitbox{4}{Rsrvd}
\bitbox{4}{MaxLgF}
\bitbox{4}{Rsrvd}
\bitbox{4}{LgFIFO}
\bitbox{7}{0}
\bitbox{9}{CLKDIV}
\end{bytefield}
\caption{CONFIG Register fields}\label{fig:CONFIG}
\end{center}\end{figure}

The CLKDIV field sets a divisor from the current clock to create a SPI clock.
The minimum value of this field is `1', corresponding to dividing the input
clock by `4'.  As discussed earlier, the input clock will be divided by
twice this field plus one.  Hence, setting this field to one will cause the
original clock to be divided by $2(1+{\tt CLKDIV})$ or $4$.  Thus an 80~MHz
input clock will become a 20~MHz SPI clock.  Attempts to set this value to
zero will be quietly ignored.

The LgFIFO field sets the log, base two, of the any memory transfer size.  The
actual transfer length will be
will be $2^{\mbox{\tiny LGFIFO}}$ bytes.  The maximum size the
device will support is returned by the {\tt MaxLgF} field, which is
currently set to $9$ for a 512~byte memory area.  This matches the current
specification, which limits sectors to only ever being 512~bytes.

To set the CONFIG register, first set the DATA register to the new config
value (or zero for the fields that will not change), and then write
{\tt 0x0ff} to the CMD register.  Likewise, to read the CONFIG register
write a {\tt 0x0bf} to the CMD register and read the CONFIG register from the
DATA register.  (Only the upper two bits of these commands are ever checked.)

\chapter{Wishbone Datasheet}\label{ch:wb}
Tbl.~\ref{tbl:wishbone}
\begin{table}[htbp]
\begin{center}
\begin{wishboneds}
Revision level of wishbone & WB B4 spec \\\hline
Type of interface & Slave, (Block/pipelined) Read/Write \\\hline
Port size & 32--bit \\\hline
Port granularity & 32--bit \\\hline
Maximum Operand Size & 32--bit \\\hline
Data transfer ordering & Big Endian \\\hline
Clock constraints & (See below)\\\hline
Signal Names & \begin{tabular}{ll}
		Signal Name & Wishbone Equivalent \\\hline
		{\tt i\_clk} & {\tt CLK\_I} \\
		{\tt i\_wb\_cyc} & {\tt CYC\_I} \\
		{\tt i\_wb\_stb} & {\tt STB\_I} \\
		{\tt i\_wb\_we} & {\tt WE\_I} \\
		{\tt i\_wb\_addr} & {\tt ADR\_I} \\
		{\tt i\_wb\_data} & {\tt DAT\_I} \\
		{\tt o\_wb\_ack} & {\tt ACK\_O} \\
		{\tt o\_wb\_stall} & {\tt STALL\_O} \\
		{\tt o\_wb\_data} & {\tt DAT\_O}
		\end{tabular}\\\hline
\end{wishboneds}
\caption{Wishbone Slave Datasheet}\label{tbl:wishbone}
\end{center}\end{table}
is required by the wishbone specification, and so it is included here.  Note
that all wishbone operations may be pipelined, to include FIFO operations,
for speed.

The particular constraint on the clock is not really a wishbone constraint, but
rather an SD--Card constraint.  Not all cards can handle clocks faster than
25~MHz.  For this reason, the wishbone clock, which forms the master clock for
this entire controller, must be divided down so that the SPI clock is within 
the limits the card can handle.

\chapter{Clocks}\label{ch:clk}

This core has been tested on a Spartan 6 using an 80~MHz system clock, as well
as on an Artix~7 using a 100~MHz system clock.

\chapter{I/O Ports}\label{ch:io}

Table.~\ref{tbl:ioports}
\begin{table}[htbp]
\begin{center}
\begin{portlist}
i\_clk & 1 & Input & Clock\\\hline\hline
i\_sd\_reset & 1 & Input & SD-Card reset, active high\\\hline\hline
i\_wb\_cyc & 1 & Output & Wishbone bus cycle active\\\hline
i\_wb\_stb & 1 & Output & Wishbone Strobe, true one clock only for each interaction\\\hline
i\_wb\_we & 1 & Output & Wishbone Write-Enable line\\\hline
i\_wb\_addr & 2 & Output & Selects our I/O register\\\hline
i\_wb\_data & 32 & Output & Incoming wishbone bus data\\\hline
o\_wb\_ack & 1 & Output & Acknowledge a WB request, always true one clock after the request\\\hline
o\_wb\_stall & 1 & Output & Always zero\\\hline
o\_wb\_data & 32 & Output & 32--bit wishbone data response\\\hline\hline
o\_cs\_n & 1 & Output & Chip--select and SPI request line\\\hline
o\_sck & 1 & Output & SD Card clock\\\hline
o\_mosi & 1 & Output & Output data wire to the SD Card\\\hline
i\_miso & 1 & Input & Input data wire from the SD Card\\\hline\hline
i\_card\_detect & 1 & Input & High if the hardware detects a card, low o.w.\\\hline\hline
o\_int & 1 & Output & An interrupt line to the CPU controller\\\hline
i\_bus\_grant & 1 & Input  & True if the SDSPI controller is controlling the bus\\\hline
o\_debug & 32 & Output & See Verilog for details\\\hline
\end{portlist}
\caption{List of IO ports}\label{tbl:ioports}
\end{center}\end{table}
lists all of the input and output ports to this core.  You may notice these
inputs and outputs are divided into sections: the master clock, the wishbone
bus, the SPI interface to the card, and three other wires.  Of these, the
last two chapters discussed the wishbone bus interface and the clock.  The
SPI interface should be fairly straightforward, so we'll move on and discuss
the other four wires.

The {\tt i\_card\_detect} wire should come from any card detection circuitry
if present.  This is an active high input.  It's used to set both the
{\tt SD\_PRESENTN} (active low) and {\tt REMOVED} bits.

This controller supports an interrupt line, {\tt o\_int}.  Upon completion of
any operation, when the SPI chip select line is deactivated (raised high),
{\tt o\_int} will be strobed for one cycle.  It is up to the logic using this
chip to catch and use that interrupt line or ignore it.  In particular, it is
possible to use that interrupt line to trigger a DMA service to move data
in or out of the FIFO, although the details of that are beyond this discussion
here.  Removing the SD card will also cause an {\tt o\_int} interrupt, but only
if the {\tt SD\_REMOVED} bit is clear.

Optionally, if the {\tt OPT\_SPI\_ARBITRATION} bit is set, then the controller
will use the {\tt i\_bus\_grant} input as feedback from a SPI bus arbiter.
This allows the controller to operate in shared SPI environments, such as when
multiple cores wish to drive the the same two wires ({\tt o\_sck} and
{\tt o\_mosi}).  When enabled, any time the controller lowers the {\tt o\_cs\_n}
line, it will then wait for {\tt i\_bus\_grant} to go high.  This is the
signal from the outside arbiter indicating that this chip has been selected
and that it is now directly driving the {\tt o\_sck} and {\tt o\_mosi} pins.
Should you not need this in your environment, you can simply leave this line
wired high or remove this logic entirely by clearing the
{\tt OPT\_SPI\_ARBITRATION} parameter.

The final bus of 32--wires, {\tt o\_debug}, is defined internally and used
when/if necessary to debug the core and watch what is going on within it.
These wires may be left unconnected in most implementations, as they are not
necessary for using the actually controller.

% Appendices
% Index
\end{document}


